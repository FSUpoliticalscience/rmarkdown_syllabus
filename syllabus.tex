\documentclass[11pt,]{article}
\usepackage[margin=1in]{geometry}
\newcommand*{\authorfont}{\fontfamily{phv}\selectfont}
\usepackage[]{mathpazo}
\usepackage{abstract}
\renewcommand{\abstractname}{}    % clear the title
\renewcommand{\absnamepos}{empty} % originally center
\newcommand{\blankline}{\quad\pagebreak[2]}

\providecommand{\tightlist}{%
  \setlength{\itemsep}{0pt}\setlength{\parskip}{0pt}} 
\usepackage{longtable,booktabs}

\usepackage{parskip}
\usepackage{titlesec}
\titlespacing\section{0pt}{12pt plus 4pt minus 2pt}{6pt plus 2pt minus 2pt}
\titlespacing\subsection{0pt}{12pt plus 4pt minus 2pt}{6pt plus 2pt minus 2pt}

\titleformat*{\subsubsection}{\normalsize\itshape}

\usepackage{titling}
\setlength{\droptitle}{-5em}

%\setlength{\parindent}{0pt}
%\setlength{\parskip}{6pt plus 2pt minus 1pt}
%\setlength{\emergencystretch}{3em}  % prevent overfull lines 

\usepackage[T1]{fontenc}
\usepackage[utf8]{inputenc}

\titleformat{\section}[hang]{\Large\bfseries\scshape}{\thesection.}{2ex}{}
\titleformat{\subsection}[hang]{\large\bfseries\scshape}{\thesection.}{2ex}{}


\usepackage{fancyhdr}
\pagestyle{fancy}
\usepackage{lastpage}
\renewcommand{\headrulewidth}{0.3pt}
\renewcommand{\footrulewidth}{0.0pt} 
\lhead{}
\chead{}
\rhead{\footnotesize International Human Rights -- Fall 2018}
\lfoot{}
\cfoot{\small \thepage/\pageref*{LastPage}}
\rfoot{}

\fancypagestyle{firststyle}
{
\renewcommand{\headrulewidth}{0pt}%
   \fancyhf{}
   \fancyfoot[C]{\small \thepage/\pageref*{LastPage}}
}

%\def\labelitemi{--}
%\usepackage{enumitem}
%\setitemize[0]{leftmargin=25pt}
%\setenumerate[0]{leftmargin=25pt}




\makeatletter
\@ifpackageloaded{hyperref}{}{%
\ifxetex
  \usepackage[setpagesize=false, % page size defined by xetex
              unicode=false, % unicode breaks when used with xetex
              xetex]{hyperref}
\else
  \usepackage[unicode=true]{hyperref}
\fi
}
\@ifpackageloaded{color}{
    \PassOptionsToPackage{usenames,dvipsnames}{color}
}{%
    \usepackage[usenames,dvipsnames]{color}
}
\makeatother
\hypersetup{breaklinks=true,
            bookmarks=true,
            pdfauthor={},
             pdfkeywords = {},  
            pdftitle={International Human Rights},
            colorlinks=true,
            citecolor=blue,
            urlcolor=blue,
            linkcolor=magenta,
            pdfborder={0 0 0}}
\urlstyle{same}  % don't use monospace font for urls


\setcounter{secnumdepth}{0}





\usepackage{setspace}

\title{\Huge\sc{International Human Rights}}
\date{Fall 2018}



\begin{document}  


		\maketitle
		
	
		\thispagestyle{firststyle}

%	\thispagestyle{empty}


	\noindent \begin{tabular*}{\textwidth}{ @{\extracolsep{\fill}} ll @{\extracolsep{\fill}}}

\hline
&\\
Professor: Kimberly R. Frugé & Course Number: INR 4075-0001 \\  
E-mail: \texttt{\href{mailto:kfruge@fsu.edu}{\nolinkurl{kfruge@fsu.edu}}} & Web: \href{http://canvas.fsu.edu}{\tt canvas.fsu.edu}\\
Office Hours: TR 10:00-noon or by appointment  &  Class Hours: MW 03:35-04:50 p.m.\\
Office: Bellamy 543  & Class Room: Bellamy 0005\\
	&  \\
	\hline
	\end{tabular*}
	
\vspace{2mm}
	


\hypertarget{course-description}{%
\section{Course Description}\label{course-description}}

For the course description: This course introduces the student to the
philosophical and legal foundations of the international human rights
regime and explores the developments of norms and institutions with
special emphasis on the post-World War II era.

In this course, students are introduced to the international human
rights regime and explores the states responsibility to protect these
rights. The course is structured to examine, who violates human rights,
why these states violate human rights, and how the domestic governments
and international organizations improve human rights standards
throughout the world.

By the end of the course students will complete the following
objectives:

\begin{enumerate}
\def\labelenumi{\arabic{enumi}.}
\item
  Explain the conceptual and historical evolution of the international
  human rights regime, and understand the empirical regularities of
  human rights violations across time and space.
\item
  Summarize academic theories about the incentives states face to
  violate human rights and to create domestic and international
  institutions protecting the rights of individuals, and determine
  whether empirical evidence supports or falsifies those theories.
\item
  Make policy recommendations regarding the protection of human rights
  based on theory and empirical evidence, and communicate those
  recommendations orally and in writing.
\end{enumerate}

\hypertarget{required-readings}{%
\section{Required Readings}\label{required-readings}}

There are no required textbooks for this course. Instead, we will
discuss a variety of readings from academic journals and excerpts from
various textbooks. You are expected to do the assigned reading
\emph{before} class. Daily reading assignments are listed on the
syllabus and can be found on the canvas course page. I will notify you
of any changes to the reading in class or by university email.

In some of the articles you will read,the methods used will be entirely
new to you. I do not expect you to familiarize yourself or understand
all the methodological tools used in the readings. Instead, it is my
hope that through readings and lecture, you will have the tools
necessary to understand, evaluate, and criticize the authors' arguments.

\hypertarget{course-evaluations}{%
\section{Course Evaluations}\label{course-evaluations}}

Your grade is a weighted average of the following components:

\begin{itemize}
\tightlist
\item
  Midterm Exam (20\%) - 100 points
\item
  Final Exam (20\%) - 100 points
\item
  Assignments (40\%) - 200 points (5 Assignments-40 points each)
\item
  Quizzes/In-class Assignments (20\%) - 100 points (10 quizzes - 10
  points each)
\end{itemize}

The grading scale for this class is as follows:

\begin{table}[h] \centering
\begin{tabular}{l  r | l r | l r   } 
Grade & Percentage  & Grade & Percentage & Grade & Percentage  \\  \hline
   &    & A     & 100 - 93\%  & A-    & 92 - 90\%   \\ 
B+    & 89 - 87\% &   B     & 86 - 83\% &   B-    & 82 - 80\%   \\  
C+    & 79 - 77\%   & C     & 76 - 73\%  & C-    & 72 - 70\%  \\ 
D+    & 69 - 67\%   & D     & 66 - 63\%   & D-    & 62 - 60\%  \\
&& F     & 59 - 0\%   &&\\ 
\end{tabular}
\end{table}

\hypertarget{exams}{%
\subsection{Exams}\label{exams}}

All students will take two exams (midterm and final), consisting of a
combination of multiple choice and short-answer questions. The exams
will be based on class lectures, assigned readings, and class
discussion.

\begin{itemize}
\tightlist
\item
  Midterm Exam:
\item
  Final Exam:
\end{itemize}

\hypertarget{assignments}{%
\subsection{Assignments}\label{assignments}}

You will complete five assignments for this course. Each assignment is
based on applying the theories from class and the readings to a real
world example. Each assignment is worth 40 points. Late assignments
receive 5 points off (one letter grade) for every day that it is late.

Assignments will be graded by (1) how well you refer to the theories
learned in class to answer the question, and (2) how well you apply
those theories to the real world example. There is no page limit. Please
write as much as you need to answer the questions and reference the
relevant theories learned in class.

\hypertarget{quizzesin-class-assignments}{%
\subsection{Quizzes/In-class
Assignments}\label{quizzesin-class-assignments}}

You will take at least 10 quizzes in this course. Some quizzes will be
graded and others are for participation (each assignment will be clearly
marked whether it is graded or for participation). Quizzes and in-class
assignments may or may not be announced prior to the day they are given.
If more than 10 quizzes are given, only the 10 highest grades will
count.

\hypertarget{course-policies}{%
\section{Course Policies}\label{course-policies}}

\begin{enumerate}
\def\labelenumi{\arabic{enumi}.}
\tightlist
\item
  \textbf{Make-Up Exams:} I have a ``no questions asked'' early exam
  policy, which means that any exam can be taken early for any reason.
  However, only exams missed due to excused absences will be eligible
  for students to take at a later date. Students who are aware that they
  will miss an exam or have missed one due to an emergency are
  responsible for contacting the instructor to arrange a new exam date
  at least a week before the exam for planned occasions and in a timely
  manner for emergencies.
\item
  \textbf{Classroom Behavior:} Students should be respectful of others
  and refrain from behaving in a disruptive manner, including
  talking/texting on cell phone, speaking out of turn, reading non-class
  material, entering (leaving) class late (early), watching netflix,
  etc. When class permits itself to discussion, students should also be
  courteous to others opinions and backgrounds. Personal attacks or
  discrimination based on race, ethnicity, gender, religion, and/or
  lifestyle will not be tolerated. If a student fails to follow any of
  these behavior guidelines they will be asked to leave, and any
  persistent behavior can result in the removal of that student from the
  course.
\item
  \textbf{Incompletes:} Incompletes will be determined on an individual
  basis and generally will only be granted in extreme cases at the
  discretion of the instructor and in consultation with the Dean of
  Students/Dean of the College of Social Sciences. Please see me as soon
  as possible to determine the correct course of action to handle any
  major situations regarding this course and/or taking an incomplete.
\item
  \textbf{Cheating:} Cheating and/or plagiarism will not be tolerated in
  this class. Any violation of the Academic Honor Policy will be
  referred to the Dean of Students and will result in a zero for the
  assignment or the course at my discretion.
\item
  \textbf{Assignment Review:} If you are concerned about your grade on
  an assignment, I am happy to review it. However, I require that you 1)
  wait 24 hours after the assignment has been returned to you to request
  review, and 2) outline your concerns about the assignment grade via
  email. Please be aware that I will regrade the entire assignment (not
  just one portion of it) if you request review, and I reserve the right
  to raise or lower your grade as a result.
\item
  \textbf{Extra Credit:} There will be no extra credit offered for this
  course.
\item
  \textbf{Email Policy:} Please include your first and last name and
  course information in the subject line of your email. When I receive
  your email, I will make every effort to respond in a timely manner,
  usually within 24 hours. Though you may receive a reply sooner than
  that, you should not expect an immediate response. Please treat all
  email correspondences with your instructor as you would treat any
  other professional exchange. Accordingly, I expect emails to be
  respectful and polite, to use correct grammar and complete sentences.
\item
  \textbf{Technology:} I can assure you, you will not find the answers
  to class discussion in your text, on facebook, twitter, or instagram.
  Cell phones and laptops are generally a distraction and detrimental to
  the classroom experience. They have the tendency to take students away
  from the lecture and impede the active engagement with students,
  instructors and/or peers. While I do permit the use of laptops and
  tablets for note-taking, I encourage you to leave your laptops at
  home, to take notes with a pen or a pencil and to engage in classroom
  discussions.
\end{enumerate}

\hypertarget{university-policies}{%
\section{University Policies}\label{university-policies}}

\hypertarget{university-attendance-policy}{%
\subsection{University Attendance
Policy:}\label{university-attendance-policy}}

Excused absences include documented illness, deaths in the family and
other documented crises, call to active military duty or jury duty,
religious holy days, and official University activities. These absences
will be accommodated in a way that does not arbitrarily penalize
students who have a valid excuse. Consideration will also be given to
students whose dependent children experience serious illness.

\hypertarget{americans-with-disabilities-act}{%
\subsection{Americans with Disabilities
Act:}\label{americans-with-disabilities-act}}

Students with disabilities needing academic accommodations should: (1)
register with and provide documentation to the Student Disability
Resource Center (SDRC); and (2) bring a letter to the instructor
indicating that you need academic accommodations and what type. This
should be done within the first week of class. This syllabus and other
class materials are available in alternative format upon request.

For more information about services available to FSU students with
disabilities, contact:\\
Student Disability Resource Center\\
874 Traditions Way\\
108 Student Services Building\\
Florida State University\\
Tallahassee, FL 32306-4167\\
(850) 644-9566 (voice)\\
(850) 644-8504 (TDD)\\
\href{mailto:sdrc@admin.fsu.edu}{\nolinkurl{sdrc@admin.fsu.edu}}\\
\url{http://www.disabilitycenter.fsu.edu}

\hypertarget{fsu-academic-honesty-code}{%
\subsection{FSU Academic Honesty Code}\label{fsu-academic-honesty-code}}

The Florida State University Academic Honor Policy outlines the
University's expectations for the integrity of students' academic work,
the procedures for resolving alleged violations of those expectations,
and the rights and responsibilities of students and faculty members
throughout the process. Students are responsible for reading the
Academic Honor Policy and for living up to their pledge to ``\ldots{}be
honest and truthful and\ldots{} {[}to{]} strive for personal and
institutional integrity at Florida State University.'' (See the Florida
State University Academic Honor Policy for more information.)

\hypertarget{syllabus-change-policy}{%
\subsection{Syllabus Change Policy:}\label{syllabus-change-policy}}

Except for changes that substantially affect implementation of the
evaluation (grading) statement, this syllabus is a guide for the course
and is subject to change with advance notice.

\hypertarget{tentative-schedule}{%
\section{Tentative Schedule}\label{tentative-schedule}}

\hypertarget{day-1-introduction}{%
\subsection{08/27 Day 1 : Introduction}\label{day-1-introduction}}

. \emph{How to Read an Academic Journal Article}. Optional.
\url{http://ugresearch.kr.edu/student/researchbytes/how-read-social-sciences-academic-journal-article}.

Miller, Steven V.
\emph{Assorted Tips for Students on Writing Research Papers}.
\url{http://svmiller.com/blog/2015/12/assorted-tips-students-research-papers/}.

\hypertarget{day-2-what-are-human-rights}{%
\subsection{08/29 Day 2 : What are Human
Rights}\label{day-2-what-are-human-rights}}

Carey, Sabine C, Mark Gibney and Steven C Poe (2010).
\emph{The politics of human rights: the quest for dignity, Chapter 1}.
Cambridge University Press.

\hypertarget{day-3-no-class}{%
\subsection{09/03 Day 3 : No Class}\label{day-3-no-class}}

Labor Day

\hypertarget{day-4-states-responsibility}{%
\subsection{09/05 Day 4 : States
Responsibility}\label{day-4-states-responsibility}}

Carey, Sabine C, Mark Gibney and Steven C Poe (2010).
\emph{The politics of human rights: the quest for dignity, Chapter 2}.
Cambridge University Press.

\hypertarget{day-5-factors-affecting-repression}{%
\subsection{09/10 Day 5 : Factors Affecting
Repression}\label{day-5-factors-affecting-repression}}

Poe, Steven C and C Neal Tate (1994). ``Repression of Human Rights to
Personal Integrity in the 1980s: A Global Analysis.'' In:
\emph{American Political Science Review} 88.04, pp.~853--872.

\hypertarget{day-6-why-repress-power-dissent}{%
\subsection{09/12 Day 6 : Why Repress-Power \&
Dissent}\label{day-6-why-repress-power-dissent}}

Ritter, Emily Hencken (2014). ``Policy Disputes, Political Survival, and
The Onset and Severity of State Repression''. In:
\emph{Journal of Conflict Resolution} 58.1, pp.~143--168.

\hypertarget{day-7-why-repress-dissent-types-of-repression}{%
\subsection{09/17 Day 7 : Why Repress-Dissent \& Types of
Repression}\label{day-7-why-repress-dissent-types-of-repression}}

Davenport, Christian, Sarah A Soule and David A Armstrong (2011).
``Protesting while black? The differential policing of American
activism, 1960 to 1990''. In: \emph{American Sociological Review} 76.1,
pp.~152--178.

Nordaas, Ragnhild and Christian Davenport (2013). ``Fight the youth:
Youth bulges and state repression''. In:
\emph{American Journal of Political Science} 57.4, pp.~926--940.

\hypertarget{day-8-why-repress-terrorism}{%
\subsection{09/19 Day 8 : Why
Repress-Terrorism}\label{day-8-why-repress-terrorism}}

Piazza, James A and James Igoe Walsh (2009). ``Transnational terror and
human rights''. In: \emph{International Studies Quarterly} 53.1,
pp.~125--148.

\hypertarget{day-9-why-repress-torture}{%
\subsection{09/24 Day 9 : Why
Repress-Torture}\label{day-9-why-repress-torture}}

Rejali, Darius (2009). \emph{Torture and democracy}. Princeton
University Press.

\hypertarget{day-10-why-repress-public-opinion-torture}{%
\subsection{09/26 Day 10 : Why Repress-Public Opinion \&
Torture}\label{day-10-why-repress-public-opinion-torture}}

Conrad, Courtenay R, Sarah E Croco, Brad T Gomez and Will H Moore
(2017). ``Threat Perception and American Support for Torture''. In:
\emph{Political Behavior}, pp.~1--21.

\hypertarget{day-11-why-repress-mass-killings}{%
\subsection{10/01 Day 11 : Why Repress-Mass
Killings}\label{day-11-why-repress-mass-killings}}

Valentino, Benjamin, Paul Huth and Dylan Balch-Lindsay (2004).
````Draining the sea'': mass killing and guerrilla warfare''. In:
\emph{International Organization} 58.2, pp.~375--407.

\hypertarget{day-12-why-repress-genocide}{%
\subsection{10/03 Day 12 : Why
Repress-Genocide}\label{day-12-why-repress-genocide}}

Yanagizawa-Drott, David (2014). ``Propaganda and conflict: Evidence from
the Rwandan genocide''. In: \emph{The Quarterly Journal of Economics}
129.4, pp.~1947--1994.

\hypertarget{day-13-why-repress-sexual-violence}{%
\subsection{10/08 Day 13 : Why Repress-Sexual
Violence}\label{day-13-why-repress-sexual-violence}}

Whitaker, Beth Elise, James Igeo Walsh and Justin Conrad (2018).
``Natural Resource Exploitation and Sexual Violence by Rebel Groups''.
In: \emph{The Journal of Politics}.

\hypertarget{day-14-why-repress-child-soldiers}{%
\subsection{10/10 Day 14 : Why Repress-Child
Soldiers}\label{day-14-why-repress-child-soldiers}}

Beber, Bernd and Christopher Blattman (2013). ``The logic of child
soldiering and coercion''. In: \emph{International Organization} 67.1,
pp.~65--104.

\hypertarget{day-15-why-repress-election-violence}{%
\subsection{10/15 Day 15 : Why Repress-Election
Violence}\label{day-15-why-repress-election-violence}}

Hafner-Burton, Emilie M, Susan D Hyde and Ryan S Jablonski (2014).
``When do governments resort to election violence?'' In:
\emph{British Journal of Political Science} 44.1, pp.~149--179.

\hypertarget{day-16-midterm-exam}{%
\subsection{10/17 Day 16 : Midterm Exam}\label{day-16-midterm-exam}}

. \emph{Midterm Exam}.

\hypertarget{day-17-stopping-repression-elections}{%
\subsection{10/22 Day 17 : Stopping
Repression-Elections}\label{day-17-stopping-repression-elections}}

Cingranelli, David and Mikhail Filippov (2010). ``Electoral Rules and
Incentives to Protect Human Rights''. In: \emph{The Journal of Politics}
72.1, pp.~243--257.

\hypertarget{day-18-stopping-represion-legislatures-veto}{%
\subsection{10/24 Day 18 : Stopping Represion-Legislatures \&
Veto}\label{day-18-stopping-represion-legislatures-veto}}

Conrad, Courtenay Ryals and Will H Moore (2010). ``What Stops the
Torture?'' In: \emph{American Journal of Political Science} 54.2,
pp.~459--476.

Rivera, Mauricio (2017). ``Authoritarian Institutions and State
Repression: The Divergent Effects of Legislatures and Opposition Parties
on Personal Integrity Rights''. In:
\emph{Journal of Conflict Resolution} 61.10, pp.~2183--2207.

\hypertarget{day-19-stopping-repression-courts-legal-institutions}{%
\subsection{10/29 Day 19 : Stopping Repression-Courts \& Legal
Institutions}\label{day-19-stopping-repression-courts-legal-institutions}}

Keith, Linda Camp (2002). ``JUDICIAL INDEPENDENCE and human rights
protection''. In: \emph{Judicature} 85.4, pp.~195--200.

Mitchell, Sara McLaughlin, Jonathan J Ring and Mary K Spellman (2013).
``Domestic legal traditions and states' human rights practices''. In:
\emph{Journal of Peace Research} 50.2, pp.~189--202.

\hypertarget{day-20-stopping-repression-domestic-economy}{%
\subsection{10/31 Day 20 : Stopping Repression-Domestic
Economy}\label{day-20-stopping-repression-domestic-economy}}

DeMeritt, Jacqueline HR and Joseph K Young (2013). ``A Political Economy
of Human Rights: Oil, Natural Gas, and State Incentives to Repress''.
In: \emph{Conflict Management and Peace Science} 30.2, pp.~99--120.

Cingranelli, David, Paola Fajardo-Heyward and Mikhail Filippov (2014).
``Principals, Agents and Human Rights''. In:
\emph{British Journal of Political Science} 44.3, pp.~605--630.

\hypertarget{day-21-stopping-repression-trade}{%
\subsection{11/05 Day 21 : Stopping
Repression-Trade}\label{day-21-stopping-repression-trade}}

Hafner-Burton, Emilie M (2005). ``Trading Human Rights: How Preferential
Trade Agreements Influence Government Repression''. In:
\emph{International Organization} 59.03, pp.~593--629.

\hypertarget{day-22-no-class}{%
\subsection{11/07 Day 22 : No Class}\label{day-22-no-class}}

Prof.~Frugé to Peace Science Society Annual Meeting

\hypertarget{day-23-no-class}{%
\subsection{11/12 Day 23 : No Class}\label{day-23-no-class}}

Veteran's Day

\hypertarget{day-24-stopping-repression-foreign-aid}{%
\subsection{11/14 Day 24 : Stopping Repression-Foreign
Aid}\label{day-24-stopping-repression-foreign-aid}}

Carnegie, Allison and Nikolay Marinov (2017). ``Foreign aid, human
rights, and democracy promotion: evidence from a natural experiment''.
In: \emph{American Journal of Political Science} 61.3, pp.~671--683.

\hypertarget{day-25-stopping-repression-naming-and-shaming}{%
\subsection{11/19 Day 25 : Stopping Repression-Naming and
Shaming}\label{day-25-stopping-repression-naming-and-shaming}}

Hafner-Burton, Emilie M (2008). ``Sticks and Stones: Naming and Shaming
the Human Rights Enforcement Problem''. In:
\emph{International Organization} 62.04, pp.~689--716.

\hypertarget{day-26-no-class}{%
\subsection{11/21 Day 26 : No Class}\label{day-26-no-class}}

Happy Thanksgiving

\hypertarget{day-27-stopping-repression-international-treaties}{%
\subsection{11/26 Day 27 : Stopping Repression-International
Treaties}\label{day-27-stopping-repression-international-treaties}}

Simmons, Beth A (2009).
\emph{Mobilizing for Human Rights: International Law in Domestic Politics}.
Cambridge University Press.

\hypertarget{day-28-stopping-repression}{%
\subsection{11/28 Day 28 : Stopping
Repression}\label{day-28-stopping-repression}}

Hill Jr, Daniel W (2010). ``Estimating the Effects of Human Rights
Treaties on State Behavior''. In: \emph{The Journal of Politics} 72.4,
pp.~1161--1174.

Welch, Ryan M (2017). ``National Human Rights Institutions: Domestic
implementation of international human rights law''. In:
\emph{Journal of Human Rights} 16.1, pp.~96--116.

\hypertarget{day-29-stopping-repression-regional-international-courts}{%
\subsection{12/03 Day 29 : Stopping Repression-Regional \& International
Courts}\label{day-29-stopping-repression-regional-international-courts}}

Sikkink, Kathryn and Hun Joon Kim (2013). ``The justice cascade: The
origins and effectiveness of prosecutions of human rights violations''.
In: \emph{Annual Review of Law and Social Science} 9, pp.~269--285.

\hypertarget{day-30-stopping-repression-icc}{%
\subsection{12/05 Day 30 : Stopping
Repression-ICC}\label{day-30-stopping-repression-icc}}

Krcmaric, Daniel (2018). ``Should I Stay or Should I Go? Leaders, Exile,
and the Dilemmas of International Justice''. In:
\emph{American Journal of Political Science} 62.2, pp.~486--498.

\hypertarget{day-31-final-exam}{%
\subsection{12/10 Day 31 : Final Exam}\label{day-31-final-exam}}

. \emph{Final Exam}.




\end{document}

\makeatletter
\def\@maketitle{%
  \newpage
%  \null
%  \vskip 2em%
%  \begin{center}%
  \let \footnote \thanks
    {\fontsize{18}{20}\selectfont\raggedright  \setlength{\parindent}{0pt} \@title \par}%
}
%\fi
\makeatother